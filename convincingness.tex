% This is samplepaper.tex, a sample chapter demonstrating the
% LLNCS macro package for Springer Computer Science proceedings;
% Version 2.20 of 2017/10/04
%
\documentclass[runningheads]{llncs}
%
\usepackage{graphicx}
\usepackage{hyperref}
\usepackage{url}
\usepackage{booktabs}


%\renewcommand\UrlFont{\color{blue}\rmfamily}

\begin{document}
%
\title{Automatic Explanation Quality Assessment in Online Learning Environments}
%
\author{Sameer Bhatnagar\inst{1} \and
Amal Zouaq\inst{1} \and
Michel C. Desmarais\inst{1} \and
Elizabeth Charles\inst{2}
}
%
\authorrunning{S. Bhatnagar et al.}

\institute{Ecole Polytechnique Montreal 
\email{\{sameer.bhatnagar,amal.zouaq,michel.desmarais\}@polymtl.ca}\and
Dawson College
\email{echarles@dawsoncollege.qc.ca}\\
}
%
\maketitle              % typeset the header of the contribution
%
\begin{abstract}
\textit{
	150-250 words
}

\keywords{Argument mining  \and Learnersourcing \and Peer Instruction}
\end{abstract}


\section{Introduction}

\section{Related Work}

\subsection{Learnersourcing \& Comparative Peer Assessment}
Ripple\cite{khosravi_ripple_2019}, AXIS\cite{williams_axis:_2016}
Juxtapeer\cite{cambre_juxtapeer:_2018}

\subsection{Argument Quality \& Convincingness}
\begin{itemize}
	\item \cite{habernal_which_2016} Predicting convincingness, reducing noise 
	in annotations by building an acyclic 
	argument graph
	\item \cite{simpson_finding_2018} Gaussian Process Preference Learning 
	\item \cite{toledo_automatic_2019} Assessment of argument quality, with a 
	dataset that has both individual scores 
	and pairwise-ranked data 
	\item \cite{gleize_are_2019} Evidence quality, predicted using a Siamese 
	network architecture 
	
\end{itemize}
  




\section{Methods}

\subsection{Data}
The dataset is comprised of pairs of student explanations for a particular 
answer choice to a given question. The first explanation is always the one 
written by the learner-annotator, while the second is an alternative which 
they either chose as more convincing, or not. The data is filtered so as 
to only keep observations where the explanations are within half a standard 
deviation in length of each other. To ensure internal reliability, we only keep 
observations where we have at least 3 
records per chosen explanation. To ensure that the explanations in each pair 
are of comparable length, we keep only those with word counts that are within 
0.5 standard deviations or 10 
words 
of each other. This leaves us a dataset with 
7159 observations, spanning 2216 learner 
annotators having worked on on average 4.0 items 
each, from a total of 109 items across three disciplines.
\begin{table}
\caption{Observations of students choosing a peer explanation as more 
	convincing than their own, or not, aggregated by discipline and whether 
	they 
	started and finished with the correct answer}

\begin{tabular}{lrrrr}
\toprule
{} &    rr &   rw &   wr &   ww \\
\midrule
Biology   &  2459 &  124 &  733 &  603 \\
Chemistry &  1151 &   51 &  228 &  236 \\
Physics   &  2288 &   66 &  278 &  334 \\
\bottomrule
\end{tabular}


\label{tab:transitions_by_discipline}
\end{table}

Table \ref{tab:transitions_by_discipline} highlights one key difference between 
the modelling task of this study, and related work in argument mining, where 
annotators are presented pairs of arguments that are always for the same 
stance, in order to limit bias due to their opinion on the motion when 
evaluating which argument is more convincing.
In a \textit{Peer Instruction} learning environment, other pairings are 
possible and pedagogically relevant. In this dataset, the majority of students 
keep the same answer choice between the two steps of the prompt, and so they 
are comparing two explanations that are either both correct (\textit{``rr''}) 
or incorrect (\textit{``wr''}). However, there is 
17\% 
of the observations in this dataset are for students who not only choose an 
explanation more convincing than their own, but also switch answer choice, 
either from the incorrect to correct, or the reverse . These pairs add a 
different level of complexity to the model, but are very pertinent in the 
pedagogical context: what are the argumentative features which can help 
students remediate an initial wrong answer choice (\textit{``wr''})? What are 
the features that might be responsible for getting studnets to actually move 
away from the correct answer choice (\textit{``rw''})?

\subsection{Models}
The first baseline model we compare to is where students simply choose the 
longer explanation of the pair, while the second is based solely on a Bag of 
Words model trained on all of the words used by students for this item, and the 
words in 


\section{Results}
\begin{table}
	\begin{tabular}{l|l|l}
		\toprule
		model & accuracy & AUC \\
		\toprule
		Argument Length &  &  \\
		SVM-R &  &  \\
		GPPL & & \\
		BiLSTM & & \\
		Siamese &  &  \\
		BERT & & \\	
	\end{tabular}
\end{table}
\section{Discussion}

\section{Future Work}

 \bibliographystyle{splncs04}
 \bibliography{MyLibrary}
\end{document}
