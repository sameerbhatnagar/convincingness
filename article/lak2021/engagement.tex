%%
%% This is file `engagement.tex',
%% The first command in your LaTeX source must be the \documentclass command.
\documentclass[sigconf]{acmart}

%%
%% \BibTeX command to typeset BibTeX logo in the docs
\AtBeginDocument{%
  \providecommand\BibTeX{{%
    \normalfont B\kern-0.5em{\scshape i\kern-0.25em b}\kern-0.8em\TeX}}}

%% Rights management information.  This information is sent to you
%% when you complete the rights form.  These commands have SAMPLE
%% values in them; it is your responsibility as an author to replace
%% the commands and values with those provided to you when you
%% complete the rights form.
\setcopyright{acmcopyright}
\copyrightyear{2021}
\acmYear{2021}
\acmDOI{10.000/0000.0000}

%% These commands are for a PROCEEDINGS abstract or paper.
\acmConference[LAK '21]{LAK '21: Learning 
Analytics \& Knowledge}{April 2021}{UCI,CA}
%\acmBooktitle{Woodstock '18: ACM Symposium on 
%Neural Gaze Detection,
%  June 03--05, 2018, Woodstock, NY}
%\acmPrice{15.00}
%\acmISBN{978-1-4503-XXXX-X/18/06}


%%
%% Submission ID.
%% Use this when submitting an article to a sponsored event. You'll
%% receive a unique submission ID from the organizers
%% of the event, and this ID should be used as the parameter to this command.
%%\acmSubmissionID{123-A56-BU3}

%% end of the preamble, start of the body of the document source.
\begin{document}

%%
%% The "title" command has an optional parameter,
%% allowing the author to define a "short title" to be used in page headers.
\title{Modelling Argument Quality in Technology-Mediated 
Peer-Instruction}

\author{Sameer Bhatnagar}
\author{Antoine Lefebvre-Brossard}
\author{Michel C. Desmarais}
\author{Amal Zouaq}
\email{{sameer.bhatnagar,antoine.lefebvre-brossard,michel.desmarais,amal.zouaq}@polymtl.ca}
\affiliation{%
  \institution{Ecole Polytechnique Montreal}
  \country{Canada}
}


\renewcommand{\shortauthors}{Bhatnagar, et al.}

%%
%% The abstract is a short summary of the work to be presented in the
%% article.
\begin{abstract}
\textbf{TO DO}
\end{abstract}

%%
%% The code below is generated by the tool at http://dl.acm.org/ccs.cfm.

\begin{CCSXML}
	<ccs2012>
	<concept>
	<concept_id>10010405.10010489.10010490</concept_id>
	<concept_desc>Applied 
	computing~Computer-assisted 
	instruction</concept_desc>
	<concept_significance>500</concept_significance>
	</concept>
	<concept>
	<concept_id>10010147.10010178.10010179</concept_id>
	<concept_desc>Computing methodologies~Natural 
	language processing</concept_desc>
	<concept_significance>500</concept_significance>
	</concept>
	</ccs2012>
\end{CCSXML}

\ccsdesc[500]{Applied computing~Computer-assisted 
instruction}
\ccsdesc[500]{Computing methodologies~Natural 
language processing}


%%
%% Keywords. The author(s) should pick words that accurately describe
%% the work being presented. Separate the keywords with commas.
\keywords{Peer Instruction, Learnersourcing}


\maketitle

\section{Introduction}

Technology-mediated peer instruction \textit{(TMPI)} platforms 
\cite{charles_harnessing_2019}\cite{univeristy_of_british_columbia_ubc/ubcpi_2019}
expand multiple choice items into a two step process.
On the first step, students must not only choose an answer choice, but also 
provide an explanation that justifies their reasoning. 
On the second step, students are prompted to revise their answer choice, by 
taking into consideration a subset of explanations written by their peers for
another answer choice.
In the case that the student wants to keep their original answer choice, but 
may be unsure of their own explanation, they are also shown peer-explanations 
for their original answer choice. 
The student now has three options:
\begin{enumerate}
	\item Change their answer choice, by indicating which of their peer's 
	explanations was most convincing
	\item keep their answer choice, but \textit{change explanations} by 
	choosing one that is for the same answer as their own
	\item choose ``I stick to my own'', indicating that their own explanation 
	is best from amongst those that are shown.
\end{enumerate}

Whenever the student goes with either of the first two scenarios above, we 
frame this as ``casting a vote'' for the chosen peer explanation.

The design and growing popularity of TMPI is inspired by three schools of 
thought: firstly, prompting students to explain their reasoning is beneficial 
to their learning \cite{chi_eliciting_1994}. 
Second, classroom based \textit{Peer Instruction}\cite{crouch_peer_2001}, often 
mediated by automated response systems (e.g. clickers), has become a prevalent, 
and often effective component in the teaching practice of instructors looking 
to drive student engagement as part of an active learning experience 
\cite{charles_beyond_2015}. 
In discussing with peers \textit{after} they have formulated their own 
reasoning, students are engaged in a higher order thinking task from Bloom's 
taxonomy, as they evaluate what is the strongest argument, before answering 
again.
Thirdly, by capturing data on which explanations students find most convincing, 
TMPI affords teachers the opportunity to mitigate the ``expert blind spot'' 
\cite{nathan_expert_2001}, addressing student misconceptions they might not 
otherwise have thought of.

In many teaching contexts, however, teachers do not have the time to provide 
feedback to every student explanation for every question item. 
The feedback students receive is primarily based on the correctness of their 
first and second answer choices, not the \textit{explanations} they write and 
choose.

Moreover, activities from online learning environments are often used for 
formative assessment, and carry little weight in terms of course credit. 
Framed as a low-stakes test, this can lead to low student motivation 
\cite{wise_low_2005}. 
The expectancy-value model \cite{pintrich_dynamic_1989}, which 
describes factors that influence the effort students will direct towards a 
task, includes ``how important they perceive the test to be'', and the 
``affective reaction to how mentally taxing the task appears to be'' 
\cite{wolf_consequence_1995}.

This makes providing feedback to students on the quality of their explanations 
a desirable goal, in order to emphasize the importance of the writing 
activity, as well as promote engagement.
The data at hand in TMPI environments enable scaling up how much feedback that 
can given. 
The ``vote'' data represent a proxy for argument quality along the dimension 
of \textit{convincingness}, as judged by peer learners. 
These votes can be aggregated into a \textit{convincingness} score to students, 
as a measure of how effective their explanations are in persuading their peers 
to change their own answer.

We set out to examine the different measures of argument quality, along the 
dimension of \textit{convincingness}, and model their role in the TMPI process. 
Our specific research questions are:
\begin{itemize}
	\item[RQ1] What factors influence whether a student will choose a peer's 
	explanation over their own in TMPI?
	\item[RQ2] What measures of argument \textit{convincingness} are most 
	useful in aggregating the ``vote'' data from TMPI?
\end{itemize}


\section{Related Work}

\subsection{Learnersourcing student explanations}
This modality is a specific case of  
\textit{learnersourcing}\cite{weir_learnersourcing_2015}, wherein students first
generate content as part of their own learning process, that is ultimately used 
to help their peers learn as well.

One of the earliest efforts to leverage peer judgments of peer-written 
explanations is from the AXIS system\cite{williams_axis:_2016}, wherein 
students solved a problem, provided an explanation for their answer, and 
evaluated explanations written by their peers.
Using a reinforcement-learning approach known as ``multi-armed bandits'', the 
system was able to select peer-written explanations that were rated as helpful 
as those written by an expert.



\subsection{Ranking Arguments for Quality}
Rank aggregation is the task of combining the preferences of multiple agents 
into a single representative ranked list.
It has long been understood that obtaining pairwise preference data may be 
less prone to error on the part of the annotator, as it is a simpler task than 
rating on scales with more gradations. 
(This is relevant in TMPI, since each student is choosing one explanation as 
the most convincing in relation to the subset of others that are shown.)
 
A classical approach for rank aggregation from pairwise preference data is 
using the Bradley-Terry model, which has been extended to incorporate the 
quality of contributions of different annotators in a crowdsourced setting when 
evaluating relative reading level in a pair passages \cite{chen_pairwise_2013}. 

When evaluating argument convincingness, one of the first approaches proposed 
is based on constructing an ``argument graph'', where a weighted edge is drawn 
from node A to node B for every pair where argument A is labelled as more 
convincing than argument B. 
After filtering example pairs that lead to cycles in the graph, PageRank scores 
are derived from this directed acyclic graph, and the PageRank 
scores of each argument are used as the gold-standard to rank for 
convincingness \cite{habernal_which_2016}.

More recently, a relatively simpler heuristic Win-Rate score has been shown to 
be competitive alternative, wherein the rank score of an argument is simply the 
(normalized) number of times that argument has been chosen as more convincing 
in a pair, divided by the number of pairs it appears in
\cite{potash_ranking_2019}.

Finally, a neural approach based on RankNet has recently yielded state of the 
art results. By joining two Bidirectional Long-Short-Term Memory Networks in a 
Siamese architecture, and appending a softmax layer to the output, 
\cite{gleize_are_2019} show that we can jointly model pairwise preferences and 
overall ranks publicly available datasets.

We will explore two of these options as part of our methodology in our rank 
aggregation step: the probabilistic Bradley-Terry model, and the simple 
heuristic scoring model. 
(We leave the neural approach for future work, as the additional work required 
to address make the models interpretable enough for the educational context is 
out of the scope of this study)


\section{Methodology}

We borrow our methodological approach from research in argument mining (AM), 
specifically related to modelling argument quality along the 
dimension of \textit{convincingness}.
A common approach is to curate pairs of arguments made in defence of the same 
stance on the same topic.
These pairs are then presented to crowd-workers who label which of the two is 
more convincing. 
These pairwise comparisons can then be aggregated using rank-aggregation 
methods so as to produce a overall ranked list of arguments.

We extend this work to the domain of TMPI, and change the prediction task: 
using only vote-data on student explanations, can we effectively predict when a 
student will choose a student explanation as more convincing than their own?

We build interpretable predictive models for this task, and inspect the 
relative feature importances in order to address our research questions from 
above.

Our feature sets include
\begin{itemize}
	\item \textit{Surface level} features, including: 
	the word count of explanation that the current student wrote; 
	word counts of explanations that were shown on the review step; 
	the number of explanations shown that were much shorter, or much longer 
	than than the student's own explanation; 
	whether the student's first answer is correct;

	\item \textit{Question and Student level} features, including: 
	the difficulty of the question; 
	the strength of the student
	(Both of these are defined by overall success rate on getting the correct 
	answer on first attempt);
	
	\item \textit{Convincingness} features, including: 
	\textbf{win-rate}, defined as the ratio of times an explanation is chosen 
	to the number of times it was shown; 
	\textbf{Bradle-Terry} score, which is the argument ``quality'' parameter 
	estimated for each explanation, according to the Bradley-Terry model, where 
	the probability of argument A being chosen over argument B is given by 

$$
P(a>b) = 
\frac{\beta_a}{\beta_a+\beta_b}
$$


$$
\ell(\boldsymbol{\beta})=\sum_{i=1}^{m}\sum_{j=1}^{m} 
[w_{ij}ln\beta_i-w_{ij}ln(\beta_i+\beta_j)]
$$

subject to $\sum_{i}\beta_i=0$.
	
	 The parameters are estimated by minimizing the log-likelihood of the 
	 explanations chosen in the 
	 pairwise comparison data
\end{itemize}

\section{Data}

\section{Results}

\section{Discussion}

\section{Limitations and Future Work}
\begin{enumerate}
	\item Students are not explicitly directed on how to evaluate their peers' 
	explanations. This may have an impact 
	https://link.springer.com/article/10.1007/s10734-017-0220-3
\end{enumerate}

\begin{acks}
\textbf{TO DO}
\end{acks}

\bibliographystyle{ACM-Reference-Format}
\bibliography{MyLibrary}

\end{document}
\endinput
%% End of file `engagement.tex'.

%%%%%%%%%%%%%%%%%%%%%%%%%%%%%%%%%
% ARCHIVE

%\subsection{Text-to-text similarity}
%Many of the features in our models are based on 
%similarity 
%metrics between 
%student explanations with each other, as well as 
%with other 
%references. 
%
%Consideration must be taken in choosing a metric 
%for measuring 
%similarity 
%between two texts. 
%The use of cosine similarity is standard practice 
%in Latent 
%Semantic 
%Indexing,
%
%\begin{equation}
%sim(T_1,T_2) = \frac{T_1 \cdot T_2}{\| T_1 \| \| 
%	T_2 \|}
%\end{equation}
%
%where $T_1$ and $T_2$ are the \textit{Tf-Idf} 
%vector 
%representations of 
%each text.
%
%However there is extensive work on text-to-text 
%similarity 
%metrics 
%\cite{mihalcea_corpus-based_2006}, 
%which can be divided into two families:
%\begin{enumerate}
%	\item Knowledge Based:
%	\item Corpus Based
%\end{enumerate}
%
%\subsection{Relative Learning Gain}
%Learning platforms used by teachers for 
%formative assessment 
%over the course of 
%a semester generate longitudinal learning 
%traces. 
%Fine grained modelling of student 
%learning with such data is 
%only possible if 
%question items have been tagged with 
%Knowledge Components\cite{corbett_knowledge_1994}, or an 
%item-skill 
%mapping\cite{barnes_q-matrix_2005} has been defined.
%Learning gains on the time-scale of a few 
%weeks, or a semester, 
%are often 
%measured using the administration of a 
%carefully validated 
%pre-post test, such 
%as the Force Concept Inventory\cite{hestenes_force_1992} from the 
%Physics Education 
%Research community.
%
%However both of these methodologies 
%require resource intensive 
%development of 
%domain knowledge mappings and validated 
%instruments. 
%In the iterative design of learning 
%tools, it is desirable to 
%have a definition 
%of learning that can be measured without 
%such prohibitive 
%limitations, even it 
%can serve as a baseline to more 
%comprehensive evaluations.
%
%We propose a definition of 
%\textit{Relative 
%	Learning Gain} for each student over 
%the course of a semester 
%which requires 
%only that all students within a group 
%have completed a common 
%set of items.
%\begin{equation}
%Relative Learning Gain_{student} = 
%\frac{W \rightarrow R_{second 
%		half of semester} }{W \rightarrow 
%	R_{first half of 
%		semester} }
%\end{equation}

%\begin{equation}
%Engagement_{student} = 
%\frac{1}{N_{explanations}}\sum{WordCount_{explanation}}
%\end{equation}


%\subsection{Vector Space Models}
%We experiment with vector space models with 
%different document representations:
%\begin{enumerate}
%	\item LSA vectors (10,50,100 components) 
%	\cite{deerwester_indexing_1990}
%	\item Glove embeddings 
%	\cite{pennington_glove:_2014}
%	\item BERT embeddings \cite{devlin_bert_2018}, 
%	out-of-the-box, and 
%	fine-tuned for the current classification task
%\end{enumerate}

%The list of features included here are derived 
%from related work in argument 
%mining 
%\cite{habernal_which_2016}\cite{persing_end--end_2016}
%on student 
%essays, automatic short answer scoring 
%\cite{mohler_text--text_2009}

%\begin{enumerate}
%	
%	\item Linguistic features
%	
%	\begin{enumerate}
%		
%		\item Surface Features
%		\begin{enumerate}
%			\item word count
%			\item sentence count
%			\item max/mean word length
%			\item max/mean sentence length
%		\end{enumerate}
%		
%		\item Lexical
%		\begin{enumerate}
%			\item uni-grams \& bigrams
%			\item Type Token Ratio
%			\item number of keywords, where 
%			keywords are defined by open-source 
%			discipline specific text-book
%			\item number of equations
%		\end{enumerate}
%		
%		\item Syntactic
%		\begin{enumerate}
%			\item POS n-grams (e.g. \textit{nouns, 
%				prepositions, 
%				verbs,conjunctions,negation, 
%				adjectives, 
%				adverbs, punctuation})
%			\item modal verbs (e.g. \textit{must, 
%				should, can, might})
%			\item contextuality/formality measure 
%			\cite{heylighen_variation_2002}
%			\item dependency tree depth
%			\item 
%		\end{enumerate}
%		
%		\item Semantic
%		The LSA vectors should be trained on 
%		domain specific corpora, such as 
%		lecture slides or a textbook in the 
%		discipline 
%		\cite{mohler_text--text_2009}. 
%		
%		
%		\begin{enumerate}
%			\item Similarity to all other 
%			explanations in LSA 
%			space
%			\item Co-reference Features 
%			\cite{persing_end--end_2016} 
%			\begin{enumerate}
%				\item Fraction of entities from 
%				the prompt mentioned in each 
%				sentence, averaged over all 
%				sentences (using neural 
%				Co-reference 
%				resolution)
%				\item Vector cosine similarity 
%				between student explanation and 
%				prompt, and answer choices 
%			\end{enumerate}
%		\end{enumerate}
%		
%		
%		\item Readability
%		\begin{enumerate}
%			\item Fleish-Kincaid
%			\item Coleman-Liau
%			\item spelling errors
%		\end{enumerate}
%	\end{enumerate}
%\end{enumerate}
%
%Features typical to NLP analyses in Learning 
%Analytics that are not included 
%here:
%\begin{enumerate}
%	\item Cohesion
%	\item Sentiment analysis
%	\item psycholinguistic features
%\end{enumerate}
