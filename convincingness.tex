% This is samplepaper.tex, a sample chapter demonstrating the
% LLNCS macro package for Springer Computer Science proceedings;
% Version 2.20 of 2017/10/04
%
\documentclass[runningheads]{llncs}
%
\usepackage{graphicx}
\usepackage{hyperref}
\usepackage{url}
\usepackage{booktabs}
\usepackage{multirow}
\usepackage{subfig}


%\renewcommand\UrlFont{\color{blue}\rmfamily}

\begin{document}
%
\title{Automatic Explanation Quality Assessment in Online Learning Environments}
\titlerunning{Convincingness \& Peer Instruction}
%
\author{Sameer Bhatnagar\inst{1} \and
Amal Zouaq\inst{1} \and
Michel C. Desmarais\inst{1} \and
Elizabeth Charles\inst{2}
}
%
\authorrunning{S. Bhatnagar et al.}

\institute{Ecole Polytechnique Montreal 
\email{\{sameer.bhatnagar,amal.zouaq,michel.desmarais\}@polymtl.ca}\and
Dawson College
\email{echarles@dawsoncollege.qc.ca}\\
}
%
\maketitle              % typeset the header of the contribution
%
\begin{abstract}
\textit{Asynchronous Peer Instruction} is an increasingly popular item type in 
online learning environments. Students first respond to a question item, but 
they must also provide an explanation for their reasoning. 
They are then presented alternative explanations as written by their peers, and 
given the opportunity to change their initial answer choice, based on those 
they find most convincing. 
The peer-explanations that students find most convincing represent valuable 
data, for teachers to better grasp their students' understanding, and for the 
learning environment itself, as higher quality explanations can be shown to 
students as examples to compare their work to. 
This study reports on the application of argument mining methods in the 
context of asynchronous peer instruction, with the objective of automatically 
identifying high quality student explanations. 
Our results offer the potential to inform the design of ``learnersourcing'' 
systems, where the content is both \textit{generated} and \textit{evaluated} by 
novices. 
These design choices are critical as these systems scale, especially with 
respect to providing pedagogically insightful reports to teachers, and 
presenting engaging alternative explanations to students to promote higher 
order thinking. 

\keywords{Argument mining  \and Learnersourcing \and Peer Instruction}
\end{abstract}


\section{Introduction}


\section{Related Work}

\subsection{Learnersourcing \& Comparative Peer Assessment}
Ripple\cite{khosravi_ripple_2019}, AXIS\cite{williams_axis:_2016}
Juxtapeer\cite{cambre_juxtapeer:_2018}

\subsection{Argument Quality \& Convincingness}
Conventional argument-mining pipelines include several successive components, 
starting with the automatic detection of argumentative units, classification of 
these units into types (e.g. major claim, minor claim, premise), and 
identification of argumentative relations (which evidence units support which 
claim). Such pipelines are essential in question-answering systems 
\cite{lippi_argumentation_2016} and are at the heart if the IBM Project Debater 
initiative. 

Work in the area of automatic evaluation of argument quality finds its roots in 
 detecting evidence in legal texts\cite{moens_automatic_2007}, but has 
 accelerated in recent years as more datasets become 
available in everyday contexts, and focus shifts to modelling more qualitative 
measures, such as \textit{convincingness}. 

Some of the earlier efforts included work on automatically 
scoring of persuasive essays \cite{persing_end--end_2016} and modelling 
persuasiveness in online debate forums \cite{tan_winning_2016}. However, 
evaluating argument \textit{convincingness} with an absolute score can be 
challenging, which has led to significant work in adopting a pairwise approach, 
where data consists of pairwise observations of two arguments, labelled with 
which of the two is most convincing.

\begin{table}
	
	\input{data/sample_obs}
	\caption{Example of instance in pairwise comparison task, where two 
	students explanations are compared, and one is chosen as more convincing}
	
	\label{tab:sample_obs}
\end{table}

In \cite{habernal_which_2016}, the authors propose a feature-rich support 
vector machine, as well as an end-to-end neural approach based on pre-trained 
Glove vectors and a bidirectional Long-Short-Term Memory network for the 
pairwise classification task. This is extended in \cite{gleize_are_2019}, where 
the authors build a Siamese network architecture, where each leg is a BiLSTM, 
taking as input the pair of explanations as Glove embeddings 
\cite{pennington_glove:_2014}, in order to detect which of argument in a pair 
has the most convincing evidence. Finally, based on the success of transformer 
models such as BERT\cite{devlin_bert_2018}, the authors of 
\cite{toledo_automatic_2019} 
release a dataset of argument pairs and show that these models accurately 
predict the most convincing argument in a pair.



\begin{itemize}

	\item \cite{toledo_automatic_2019} Assessment of argument quality, with a 
	dataset that has both individual scores 
	and pairwise-ranked data 
	
\end{itemize}
  




\section{Methods}


\begin{table}
\begin{tabular}{ |l|l|l|}
	\hline
	Feature Type & Feature & $\tau$ \\
	\hline
	\multirow{6}{*}{Lexical} 
	& Uni+BiGrams & \\
	& Spelling Errors &  \\
	& Equations & \\
	& Type Token Ratio &  \\
	& Punctuation &\\
	& Readability scores&\\
	\hline
	\multirow{5}{*}{Syntactic} 
	& PoS Uni+Bigrams &   \\
	& Dependancy Tree Depth  & \\
	& Conjunctions  & \\ 
	& Modal Verbs & \\ 
	& Tree Production Rules &\\
	\hline
	\multirow{3}{*}{Semantic} 
	& LSA Similarity &  \\
	& LSA Similarity Question & \\
	& Likelihood (Textbook) & \\ 
	\hline
\end{tabular}

\caption{Features used in experiments with Linear SVM, annotated with 
	kendall $\tau$ correlation and sigificance with target label}
\label{tab:features}
\end{table}

\subsection{Data}
The dataset is comprised of pairs of student explanations for a particular 
answer choice to a given question. The first explanation is always the one 
written by the learner-annotator, while the second is an alternative which 
they either chose as more convincing, or not. The data is filtered so as 
to only keep observations where the explanations are within half a standard 
deviation in length of each other. To ensure internal reliability, we only keep 
explanations that were chosen at least \input{data/VOTES_MIN}times. To ensure 
that the explanations in each pair 
are of comparable length, we keep only those with word counts that are within 
\input{data/STD_FRAC}standard deviations or 
\input{data/MAX_WORD_COUNT_DIFF}words of each other. This leaves us a dataset 
with \input{data/N}observations, spanning \input{data/n_students}learner 
annotators having completed, on average, \input{data/avg_q_per_student}items 
each, from a total of \input{data/n_questions}items across three disciplines, 
with at least \input{data/MIN_RECORDS_PER_QUESTION}explanation-pairs per item.
\begin{table}
\caption{Observations of students choosing a peer explanation as 
more 
	convincing than their own, or not, aggregated by discipline and whether 
	they 
	started and finished with the correct answer}

\begin{tabular}{llr}
\toprule
        &    &     N \\
discipline & transition &       \\
\midrule
Biology & rr &  1800 \\
        & wr &   403 \\
        & rw &   121 \\
Chemistry & rr &  1232 \\
        & wr &   390 \\
        & rw &    82 \\
Physics & rr &  1479 \\
        & wr &   358 \\
        & rw &   103 \\
\bottomrule
\end{tabular}


\label{tab:transitions_by_discipline}
\end{table}

Table \ref{tab:transitions_by_discipline} highlights one key difference between 
the modelling task of this study, and related work in argument mining, where 
annotators are presented pairs of arguments that are always for the same 
stance, in order to limit bias due to their opinion on the motion when 
evaluating which argument is more convincing.
In a \textit{Peer Instruction} learning environment, other pairings are 
possible and pedagogically relevant. In this dataset, the majority of students 
keep the same answer choice between the two steps of the prompt, and so they 
are comparing two explanations that are either both correct (\textit{``rr''}) 
or incorrect (\textit{``wr''}). However, there is 
\input{data/frac_switch}\% of the observations in this dataset are for students 
who not only choose an explanation more convincing than their own, but also 
switch answer choice, 
either from the incorrect to correct, or the reverse . These pairs add a 
different level of complexity to the model, but are very pertinent in the 
pedagogical context: what are the argumentative features which can help 
students remediate an initial wrong answer choice (\textit{``wr''})? What are 
the features that might be responsible for getting students to actually move 
away from the correct answer choice (\textit{``rw''})?

\subsection{Models}


\begin{figure}
	\subfloat[Baseline models on myDalite]
	{	
	\begin{tabular}{llr}
\toprule
          model &   scoring &  value \\
\midrule
     ArgLongest &  accuracy &   0.59 \\
      ArgBoWGen &   roc\_auc &   0.77 \\
 ArgBoWItemSpec &   roc\_auc &   0.78 \\
      ArgBoWGen &  accuracy &   0.72 \\
 ArgBoWItemSpec &  accuracy &   0.72 \\
\bottomrule
\end{tabular}

    }\hfill
	\subfloat[Baselines on IBM argument ranking dataset]
	{	
	\begin{tabular}{llr}
\toprule
          model &   scoring &  value \\
\midrule
     ArgLongest &  accuracy &   0.53 \\
 ArgBoWItemSpec &   roc\_auc &   0.90 \\
 ArgBoWItemSpec &  accuracy &   0.82 \\
\bottomrule
\end{tabular}

    }
	\label{tab:baselines}
\end{figure}


In pairwise classification tasks so 50\% is the baseline performance.  
In Table \ref{tab:baselines} we begin by comparing the \textit{ArgLongest} 
model baseline, where we predict that students simply choose the longer 
explanation of the pair.
We also include two baseline models on \textit{Bag of Words} models: 
\textit{ArgBoWGen} where the term-document-matrix is built from an open-source 
textbook from the corresponding discipline\footnote{https://openstax.org/}, and 
\textit{ArgBoWItemSpec}, where the term-document matrix is built from the words 
students have used for the item (pertinent when no reference text is available 
for a discipline).
As this is a new context for these argument mining methods, we include the same 
baselines on the carefully curated \textit{IBMArgPair} dataset, which is of the 
same format.



For our experiments, we begin by following the line of work proposed by 
\cite{habernal_which_2016}, and experiment with a feature-rich linear SVM 
classifier for the pairwise classification task. We use a similar feature set, 
which we categorize as \textbf{lexical}, \textbf{syntactic}, and 
\textbf{semantic}, as described in Table \ref{tab:features}. we begin by 
computing the feature vector for each explanation, and compute the difference 
for each pairwise ranking instance as per the well established SVM-Rank 
algorithms \cite{joachims_optimizing_2002}, training the model to learn which 
of the pair is the more convincing argument. 


\section{Results}
\begin{figure}
	\subfloat[BERT model, fine tuned on myDalite data, by discipline]
	{	
		\input{data/BERT_dalite}
	}\hfill
	\subfloat[BERT model performance on Biology data, by answer transition]
	{	
		\input{data/BERT_Bio_transition}
	}
	\label{tab:BERT_dalite}
\end{figure}

\begin{table}
	\begin{tabular}{l|l|l}
		\toprule
		model & accuracy & AUC \\
		\toprule
		SVM &  &  \\
		BERT & & \\	
	\end{tabular}
\end{table}
\section{Discussion}

\section{Future Work}
In this study we do not ever infer which are, overall, the most convincing 
student explanations for any given item. Inferring a gold standard of global 
rankings, starting from these pairwise preference data can be accomplished 
using research from the information retrieval 
community\cite{chen_pairwise_2013}. Work on deriving point wise scores for 
argument pairs is proposed as a Gaussian Process Preference Learning task by 
\cite{simpson_finding_2018}. Seeing the lack of pointwise labels for overall 
convincingness, \cite{toledo_automatic_2019} released a dataset where they 
collect this data as well. A comparable source of data inside the myDALITE 
platform are the feedback scores teachers can optionally provide to students on 
thier explanations.


 \bibliographystyle{splncs04}
 \bibliography{MyLibrary}
\end{document}
